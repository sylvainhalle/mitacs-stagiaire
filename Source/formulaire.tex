%% À compiler avec lualatex
\documentclass{mitacs-stagiaire}

\begin{document}
%% Renseignez les paramètres dans le fichier ci-dessous
\newcommand{\titreprojet}{Applications du convecteur temporel}
\newcommand{\itprojet}{ITXXXX}
\newcommand{\prenomstagiaire}{Marty}
\newcommand{\nomstagiaire}{McFly}
\newcommand{\auteur}{Emmett Brown}
\newcommand{\partenaire}{Fusion inc.}
\newcommand{\cosuperviseur}{Biff Tannen}

\mitacsheader

\title{Proposition de recherche Accélération}
\author{\auteur}
\date{}
\maketitle
\hypersetup{
  pdftitle = {Proposition de recherche Accélération},
  pdfauthor = {\auteur}
}

\begin{center}
\color{darkred}\bfseries\itshape Les formulaires dûment remplis doivent être soumis à grants-subventions@mitacs.ca en formats PDF et Word. Veuillez indiquer le numéro d'identification du projet (IT) dans la ligne d'objet.
\end{center}
%
{\small\itshape
\begin{itemize}[label=$\circ$]
\item Pour un traitement plus rapide de vos informations, veuillez conserver la structure et le format de ce formulaire.
\item Ne modifiez aucune ligne ni aucun tableau.
\item \bfseries Toutes les sections portant un \og{}{\color{red}*}\fg{} sont {\color{darkred} \itshape OBLIGATOIRES}.
\end{itemize}
}

\section{Renseignements sur le projet\oblig}

\begin{tabdonnees}
\tabitem{Titre du projet} & \titreprojet \\
\tabitem{No d'identification du projet (IT)} & \itprojet \\
\tabitem{Professeur·e superviseur·e canadien·ne {\color{gray}(exclure les titres)}} & \auteur \\
\tabitem{Professeur·e superviseur·e canadien·ne — courriel (utilisé pour les projets Mitacs)} & \url{\auteuremail}\\
\tabitem{Co-superviseur} & \cosuperviseur \\
\tabitem{Organisation(s) partenaire(s) :} & \partenaire \\
\end{tabdonnees}

\section{Renseignements sur le stage\oblig}

\begin{tabdonnees}
\tabitem{Date de début du stage} & \datedebut \\
\tabitem{Nombre total de stages} & \nombretotal \\
\tabitem{Durée de chaque stage (mois)} & \dureestage \\
\tabitem{Les unités de stage seront-elles consécutives?} & Oui \vskip 1pt Si NON, remplissez l'{\color{darkred}Annexe C} \\
\tabitem{Montant de l'allocation par unité de stage \tispace
{\normalfont REMARQUE: le montant de l'allocation minimum est de 10~000~\$ par unité de stage pour Accélération}} & \montantstage \\
\tabitem{Est-ce que la nouvelle ou le nouveau stagiaire remplace une ou un stagiaire déjà affecté au projet?} & Non \\
\tabtabitem{a. Nom de la ou du stagiaire à remplacer} & \\
\tabtabitem{b. Dernière journée de travail de la ou du stagiaire à remplacer} & \\
\end{tabdonnees}

\subsection{Rôle de la nouvelle ou du nouveau stagiaire dans le projet}

\subsubsection{Établissez un lien entre la nouvelle ou le nouveau stagiaire et les objectifs précis qui lui ont été attribués dans la demande originale. Il faut bien distinguer les tâches de la demande originale qui ont été attribuées à la nouvelle ou au nouveau stagiaire.}

% Établissez un lien entre la nouvelle ou le nouveau stagiaire et les objectifs précis qui lui ont été attribués dans la demande originale. Il faut bien distinguer les tâches de la demande originale qui ont été attribuées à la nouvelle ou au nouveau stagiaire.}

%% :wrap=soft:

\subsubsection{S'il y a lieu, précisez les changements qui ont été apportés aux activités de la nouvelle ou du nouveau stagiaire, son niveau d'études, son département ou son expérience de recherche par rapport à la proposition originale. Décrivez brièvement les changements et expliquez pourquoi cette ou ce stagiaire convient au projet (p. ex., décrivez toute modification apportée au projet et tout soutien supplémentaire fourni pour s'assurer de réaliser le projet suivant le niveau d'études de la ou du stagiaire).}

% S'il y a lieu, précisez les changements qui ont été apportés aux activités de la nouvelle ou du nouveau stagiaire, son niveau d'études, son département ou son expérience de recherche par rapport à la proposition originale. Décrivez brièvement les changements et expliquez pourquoi cette ou ce stagiaire convient au projet (p. ex., décrivez toute modification apportée au projet et tout soutien supplémentaire fourni pour s'assurer de réaliser le projet suivant le niveau d'études de la ou du stagiaire).



%% :wrap=soft:

\section{Renseignements sur le ou la stagiaire\oblig}

\begin{tabdonnees}
\tabitem{Nom} & \prenomstagiaire{} \nomstagiaire{} \\
\tabitem{Niveau d'études pendant le stage} & \niveauetudes \\
\tabitem{Date prévue d'obtention du diplôme 
(S'il s'agit d'une chercheuse ou d'un chercheur au postdoctorat, indiquez la date d'obtention du doctorat)} & \datediplome \\
\tabitem{Établissement d'enseignement fréquenté pendant le stage} & \etablissement \\
\tabitem{Département (Nom officiel du département requis)} & \departement \\
\tabitem{Téléphone principal} & \telephoneprincipal \\
\tabtabitem{a. Autre téléphone ou téléphone cellulaire} & \\
\tabitem{Courriel permanent} & \stagiaireemail \\
\tabtabitem{a. Autre adresse courriel} & \\
\tabitem{Citoyenneté} & \citoyennete \\
\tabnoitem{Si Citoyenneté d'un autre pays, veuillez préciser} & \citoyennetepays \\
\tabitem{Est-ce que la ou le stagiaire a atteint l'âge légal au Canada {\color{gray} (18 ans ou plus)}?} & Oui \\
\tabitem{Est-ce que le ou la stagiaire réalisera des unités de stage au sein d'une organisation partenaire à l'extérieur du Canada?} & Non \\
\tabtabitem{a. Durée à l'étranger} & N/A \\
\tabtabitem{b La personne stagiaire changera-t-elle de programme ou statut pendant le stage? \color{gray}
Par exemple, de la maîtrise au doctorat, ou d'un statut « aux études » de baccalauréat à « diplômé·e »?} & Non \\
\end{tabdonnees}

\textbf{\small\color{darkred} La collecte de ces données constitue une exigence pour nos bailleurs de fonds, et elle contribue à assurer un financement constant pour nos programmes.}


\subsection{Conflit d'intérêts: Est-ce que la ou le stagiaire a (ou a déjà eu) un lien ou est (ou a déjà été) dans une position de propriété, d'influence, d'emploi ou y a-t-il toute autre circonstance relativement à l'organisation partenaire ou à d'autres participantes et participants du programme qui pourrait contribuer à un conflit d'intérêts ou à l'apparence d'un conflit d'intérêts? Veuillez consulter la Politique sur les conflits d'intérêts de Mitacs.}

Non

{\bfseries
Si vous avez répondu \og{}OUI\fg{} à l'une des questions ci-dessus, veuillez remplir le Formulaire de déclaration de conflit d'intérêts et d'admissibilité du ou de la stagiaire de Mitacs et l'envoyer à votre conseiller ou conseillère Mitacs aux fins d'examen AVANT de soumettre ce document. Si vous faites une demande au programme Accélération Entrepreneur, veuillez remplir le formulaire de déclaration de CI Accélération Entrepreneur de Mitacs.
}

\subsection{Politique sur la recherche en technologies sensibles et sur les affiliations préoccupantes:
Êtes-vous actuellement affilié·e à l'une des organisations de recherche nommées répertoriées ou bénéficiez-vous d'un financement ou d'un soutien en nature de la part de l'une d'entre elles?}

Non

{\bfseries
Toute partie actuellement affiliée à une organisation de recherche nommée ou qui reçoit du financement ou du soutien en nature de celle-ci n'est pas admissible pour participer à un projet financé par Mitacs qui implique de la recherche visant à faire avancer un domaine de recherche en technologies sensibles. Mitacs examinera également un échantillon de demandes, sélectionné à sa seule discrétion, et validera l'exactitude des déclarations remplies.

Mitacs se réserve le droit de refuser le financement, à tout moment, d'un projet qui fait progresser un domaine de recherche en technologies sensibles en raison d'affiliations préoccupantes.
}

\newpage

\newcommand{\addul}[1]{\underline{#1}}
\titleformat{\section}{\centering\Large\bfseries}{\thesection}{1em}{\addul}
\stepcounter{section}
\section*{4. Protocole d'entente Mitacs}
{
\bfseries
Les participantes et participants dont le nom apparaît ci-dessous confirment que les renseignements présentés sont exacts et qu'ils reflètent leur intention de déposer une demande au programme Mitacs Accélération. Ils et elles acceptent également d'effectuer un stage portant sur la demande ci-jointe. Les participantes et participants reconnaissent avoir lu, compris et accepté d'observer et de soutenir les responsabilités du projet qui les visent, lesquelles peuvent être consultées au \url{https://www.mitacs.ca/fr/programmes/acceleration/responsabilites-projet} et comprennent, sans s'y limiter, ce qui suit: il est entendu que la contribution de l'organisation partenaire sera versée à Mitacs inc. en dollars canadiens avant le début du stage; dans l'éventualité où la contribution de l'organisation partenaire est à l'établissement d'enseignement, l'établissement d'enseignement fera suivre ces fonds à Mitacs. À la suite de l'approbation de la recherche du projet et de la réception des fonds de l'organisation partenaire par Mitacs, Mitacs versera les fonds à l'établissement d'enseignement canadien à titre de subvention de recherche à la professeure superviseure canadienne ou au professeur superviseur canadien et une allocation/un salaire sera versé à l'étudiant·e par l'établissement d'enseignement à partir de cette subvention. Les dépenses associées à cette demande, telles que décrites dans le budget, doivent seulement être engagées après l'approbation de la recherche et la réception des fonds de l'organisation partenaire par Mitacs.

Mitacs n'assume aucune responsabilité à l'égard de toute perte, y compris, sans s'y limiter, à des accidents, des maladies, des déplacements ou autres pertes qui peuvent survenir pendant la période de stage. Chacune des parties signataires convient qu'il lui incombe de s'assurer qu'elle dispose d'une assurance appropriée et qu'elle répond aux politiques institutionnelles concernant les exigences en matière de santé et de sécurité ainsi que toute autre préparation requise avant d'entreprendre un voyage. Les parties conviennent également que la ou le stagiaire devra produire un rapport de fin de projet et que tous les participants répondront à un sondage de fin de projet qui sera remis à Mitacs au plus tard un mois après la fin du projet.

\textit{Pour les projets qui comprennent un déplacement international:} En reconnaissant que les séjours à l'étranger peuvent grandement enrichir les connaissances et l'acquisition d'expérience d'un stagiaire, Mitacs approuvera les déplacements à l'étranger sous réserve que ceux-ci ne mettent pas en danger la sécurité des stagiaires et que les politiques de l'établissement d'enseignement d'attache soient respectées. En signant le présent accord, vous reconnaissez que l'établissement d'enseignement d'attache accepte de s'engager à aider la ou le stagiaire dans ses démarches visant à satisfaire toutes les exigences de l'établissement d'enseignement ayant trait à la recherche à l'étranger et que la ou le stagiaire comprend sa responsabilité de souscrire une assurance maladie appropriée pour sa destination. Les participants aux projets qui comprennent des déplacements internationaux reconnaissent que des responsabilités du projet supplémentaires s'appliquent à chacun d'eux qui peuvent être consultées au \url{https://www.mitacs.ca/fr-ca/nos-programmes/acceleration/}. Les participants à des projets qui comprennent des déplacements internationaux reconnaissent également que Mitacs sera dans l'impossibilité de verser les fonds et que le stage ne pourra pas commencer tant que le formulaire prédépart international et le code de déontologie signés n'auront pas été reçus.

Toutes les parties participant au programme Mitacs Accélération sont tenues de respecter les règlements standards sur la propriété intellectuelle (PI) établis par l'établissement d'enseignement où la ou le stagiaire est inscrit·e, à moins qu'un accord séparé, valide pendant la durée du stage, soit négocié entre le ou les établissements d'enseignement et l'organisation partenaire. Si vous avez des ententes séparées relatives à la PI entre vous et l'établissement d'enseignement, vous reconnaissez en signant ce protocole d'entente être assujetti·e à leurs conditions précises.  Dans le cas où vous n'avez aucune entente séparée, vous êtes assujetti·e aux conditions standards de PI de l'établissement d'enseignement et en signant cette entente, vous acceptez les conditions de l'établissement d'enseignement où la ou le stagiaire est inscrit·e. Les politiques en matière de PI propres à chaque établissement en ce qui concerne les stages Accélération se trouvent à la page \url{https://www.mitacs.ca/fr/programmes/acceleration/faq}.

Les participants acceptent également que Mitacs affiche le titre du projet, l'aperçu du projet pour le public, le nom du ou des organisations partenaires, le nom de la ou du stagiaire ou des stagiaires, le nom de la ou du superviseur·e ou des superviseur·es et l'établissement d'enseignement participant sur la page www.mitacs.ca/fr/projets. Ces renseignements pourraient aussi être utilisés par Mitacs pour faire la publicité du programme Mitacs Accélération. La Politique de confidentialité de Mitacs est disponible sur la page suivante: \url{https://www.mitacs.ca/fr/declaration-de-confidentialite}.

Les participantes et participants au stage (stagiaire, professeur·e superviseur·e et organisation partenaire) acceptent aussi le ou les addendas suivants:

Mitacs ne requiert, ne vérifie, ni n'impose aucune autre condition que celles indiquées par les participantes et participants dans le ou les addendas ci-dessus.
}

\rule{\linewidth}{2pt}

%%:wrap=soft:

\subsection{Titre du projet \\                                                        
\rule{24pt}{0pt}\color{gray}(de la proposition originale)}
        
\titreprojet

\subsection{Signatures des participantes et participants\oblig\\ 
\rule{24pt}{0pt}\color{gray}(Les signatures physiques, électroniques et les images doivent se limiter à la case fournie)}

\subsubsection{Stagiaire}
\begin{tabsignatures}
\tabsigitem{Nom:} & \multicolumn{2}{l:}{\prenomstagiaire{} \nomstagiaire{}} \\
\tabsigitem{} & \multicolumn{2}{p{4.5in}:}{Pour les stagiaires participant au programme Parcours Autochtones:
\tispace
$\boxempty$ Le ou la stagiaire s'identifie comme Autochtone.} \\
\tabsigitem{Signature:} & \signaturestagiaire & \textbf{Date:} \datesigstagiaire \\
\end{tabsignatures}

\subsubsection{Professeur·e superviseur·e au Canada} 
\begin{tabsignatures}
\tabsigitem{Nom:} & \multicolumn{2}{l:}{\auteur} \\
\tabsigitem{Signature:} & \signaturesuperviseur & \textbf{Date:} \datesigsuperviseur \\
\end{tabsignatures}             

\subsubsection{Organisation partenaire}
\begin{tabsignatures}
\tabsigitem{Nom:} & \multicolumn{2}{l:}{\sigpartenaire} \\
\tabsigitem{Dénomination sociale de l'organisation:} & \multicolumn{2}{l}{\partenaire} \\
\tabsigitem{} & \multicolumn{2}{p{4.5in}:}{L'organisme partenaire s'engage à verser la contribution financière indiquée dans la demande initiale. En signant, l'organisme partenaire accepte l'intégration du nouveau stagiaire au projet.} \\
\tabsigitem{Signature:} & \signaturepartenaire & \textbf{Date:} \datesigpartenaire \\
\end{tabsignatures}

\subsubsection{Signatures additionnelles (si applicable)}
\begin{tabsignatures}
\tabsigitem{Nom:} & \multicolumn{2}{l:}{} \\
\tabsigitem{Rôle:} & & \textbf{Org:} \\
\tabsigitem{Signature:} &  & \textbf{Date:} \\
\end{tabsignatures}

\begin{tabsignatures}
\tabsigitem{Nom:} & \multicolumn{2}{l:}{} \\
\tabsigitem{Rôle:} & & \textbf{Org:} \\
\tabsigitem{Signature:} &  & \textbf{Date:} \\
\end{tabsignatures}

\begin{tabsignatures}
\tabsigitem{Nom:} & \multicolumn{2}{l:}{} \\
\tabsigitem{Rôle:} & & \textbf{Org:} \\
\tabsigitem{Signature:} &  & \textbf{Date:} \\
\end{tabsignatures}

\newpage
\section*{Annexe A – Formulaire de consentement de la ou du stagiaire Accélération\oblig}

{
\bfseries
\begin{center}
\MakeUppercase{Utilisation et divulgation des renseignements personnels fournis à Mitacs}
\end{center}

\begin{enumerate}
\item Tout renseignement personnel recueilli est assujetti à la législation relative à la protection des renseignements personnels et à la Politique de protection des renseignements personnels de Mitacs pour les participantes et participants aux programmes. Pour une description de l'engagement de Mitacs à protéger les renseignements personnels fournis par les demandeurs aux programmes, veuillez consulter \url{https://www.mitacs.ca/fr/declaration-de-confidentialite}.
\item Tous les renseignements fournis dans cette demande seront mis à la disposition du personnel de Mitacs chargé de gérer la demande, pour les activités incluant l'identification des évaluatrices et évaluateurs pairs appropriés, la gestion et le contrôle des bourses, la compilation de statistiques et l'évaluation du programme.
\item Les renseignements fournis dans cette demande seront mis à la disposition des évaluatrices et évaluateurs internes et externes qui sont des personnes expertes recrutées dans les secteurs postsecondaire, public et privé. Les évaluatrices et évaluateurs doivent tous s'engager à conserver les renseignements des demandes confidentiels.
\item Les coordonnées figurant dans cette demande peuvent être utilisées par le personnel de Mitacs afin de vous contacter à l'avenir pour:
\begin{enumerate}
\item des invitations à faire l'objet d'histoires ou d'articles, à intervenir ou à participer à des événements, à fournir un témoignage de votre expérience ou un billet de blogue;
\item vous informer par rapport à des occasions pour les anciennes et anciens de Mitacs;
\item participer à des sondages de recherche pour les anciennes et anciens de Mitacs.
\end{enumerate}
Vous aurez la possibilité de vous désabonner des courriels qui vous sont envoyés lorsque tous les engagements par rapport à la recherche qui fait l'objet de cette demande seront remplis.
\item Votre nom, établissement d'enseignement et département ainsi que le titre de votre projet peuvent être fournis aux bailleurs de fonds fédéraux, provinciaux/territoriaux et de l'établissement d'enseignement du programme Accélération pour:
\begin{enumerate}
\item permettre à Mitacs de rendre compte de ses engagements par rapport aux ententes de financement;
\item permettre aux bailleurs de fonds d'évaluer le programme.
\end{enumerate}
Des renseignements supplémentaires, comme le numéro de passeport et la date de naissance, peuvent être fournis aux bailleurs de fonds étrangers du programme, s'il y a lieu, à des fins d'examen et de production de rapports. 
\item Votre nom, vos coordonnées et autres renseignements personnels, au besoin, peuvent être fournis aux établissements d'enseignement participant au stage pour leur permettre de gérer la bourse, d'approuver le Formulaire prédépart international, s'il y a lieu, et de produire des rapports.
\end{enumerate}

Je, soussigné(e), CONSENS par la présente à l'utilisation et à la divulgation des informations contenues dans ma demande aux fins décrites ci-dessus.
}
%% :wrap=soft:

\begin{center}
{\arrayrulecolor{black!20}
\begin{tabular}{:c:c:c:}
\hdashline
\prenomstagiaire{} \nomstagiaire{} & \signaturestagiaire & \datesigstagiaire \\
\hline
Nom du/de la stagiaire & \rule{1cm}{0pt} Signature \rule{1cm}{0pt} & \rule{1cm}{0pt} Signature Date \rule{1cm}{0pt} \\
\hdashline
\end{tabular}
}
\end{center}

\newpage

\section*{Annexe B – Modèle de CV de la ou du stagiaire\oblig}

\begin{center}
\small
Toutes les sections portant un \og{}{\color{red}*}\fg{} sont {\color{darkred} OBLIGATOIRES}
\end{center}

\arrayrulecolor{black!90}
\begin{tabular}{|p{2.825in}|p{2.825in}|}
\hline
\multicolumn{2}{|l|}{\textbf{Date de présentation:}\oblig: \datesigstagiaire} \\
\hline
\multicolumn{2}{|l|}{\textbf{\No{} IT du projet:}\oblig: \itprojet} \\
\hline
\textbf{Nom de famille:} \nomstagiaire & \textbf{Prénom:} \prenomstagiaire \\
\hline
\end{tabular}

\begin{tabularx}{6in}{|X|X|X|X|X|}
\hline
\rowcolor{black!30}
\multicolumn{5}{|p{5.82in}|}{\textbf{ÉTUDES POSTSECONDAIRES}
\tispace\small
Remarque: La date de fin attendue du programme d'études doit être documentée pour vérifier le statut à temps plein de l'étudiante ou de l'étudiant et son admissibilité au projet.\oblig} \\
\hline
\rowcolor{black!30} Diplôme\oblig
\tispace
Énumérez l'affiliation postsecondaire actuelle ou la plus récente* jusqu'à la plus ancienne.
&
Établissement d'enseignement\oblig & Département \oblig & Lieu \oblig & Date de début et de fin\oblig
\tispace
(mm/aaaa à mm/aaaa) \\
\hline
\rowcolor{white}
\ligneAdiplome & \ligneAetablissement & \ligneAdepartement & \ligneAlieu & \ligneAdates \\
\hline
\ligneBdiplome & \ligneBetablissement & \ligneBdepartement & \ligneBlieu & \ligneBdates \\
\hline
\ligneCdiplome & \ligneCetablissement & \ligneCdepartement & \ligneClieu & \ligneCdates \\
\hline
\ligneDdiplome & \ligneDetablissement & \ligneDdepartement & \ligneDlieu & \ligneDdates \\
\hline
\ligneEdiplome & \ligneEetablissement & \ligneEdepartement & \ligneElieu & \ligneEdates \\
\hline
\end{tabularx}

\begin{tabularx}{6in}{|X|X|X|}
\hline
\rowcolor{black!30}
\multicolumn{3}{|p{5.82in}|}{\textbf{EXPÉRIENCE - EN MILIEU POSTSECONDAIRE, EN RECHERCHE, EN ENTREPRISE}
\tispace\small
\oblig Assurez-vous d'inclure toute affiliation avec l'organisation partenaire associée au \no{} IT lié au stage.} \\
\hline
\rowcolor{black!30} Titre du poste occupé
&
Nom de l'organisation, département et lieu
\tispace (ex.\ Exemple d'entreprise, Exemple d'équipe - Ville, Province) 
&
Durée du poste
\tispace
(mm/aaaa à mm/aaaa) \\
\hline
\rowcolor{white}
\ligneAposte & \ligneAorganisation & \ligneAduree \\
\hline
\ligneBposte & \ligneBorganisation & \ligneBduree \\
\hline
\ligneCposte & \ligneCorganisation & \ligneCduree \\
\hline
\end{tabularx}

\newpage

\section*{Annexe C - Date du stage et tableau de financement}

\begin{center}\bfseries
REMARQUE: la date de début proposée de tout stage doit être postérieure à la date de réception de la Lettre de résultat.
\end{center}

\begin{tabularx}{6in}{|X|X|X|X|}
\hline
\rowcolor[HTML]{C0C0C0} 
\multicolumn{1}{|c|}{\cellcolor[HTML]{C0C0C0}}                                          & \multicolumn{2}{c|}{\cellcolor[HTML]{C0C0C0}\textbf{Période des unités de stage}}                                                                                                                                                                                     & \multicolumn{1}{c|}{\cellcolor[HTML]{C0C0C0}}                                                   \\ \cline{2-3}
\rowcolor[HTML]{C0C0C0} 
\multicolumn{1}{|c|}{\multirow{-2}{*}{\cellcolor[HTML]{C0C0C0}\textbf{Unité de stage}}} & \multicolumn{1}{c|}{\cellcolor[HTML]{C0C0C0}\begin{tabular}[c]{@{}c@{}}Unité de stage\\ Date de début\\ (JJ/MM/AAAA)\end{tabular}} & \multicolumn{1}{c|}{\cellcolor[HTML]{C0C0C0}\begin{tabular}[c]{@{}c@{}}Unité de stage\\ Date de fin\\ (JJ/MM/AAAA)\end{tabular}} & \multicolumn{1}{c|}{\multirow{-2}{*}{\cellcolor[HTML]{C0C0C0}\textbf{Montant de l'allocation}}} \\ 

                                                                                        \hline

\directlua{dofile("financement.lua")} % runs every time                                                                                              
\end{tabularx}

\bfseries{
S'il y a lieu: \newline
Mettez à jour la feuille de calcul Excel (Plan de ressources Accélération) si des changements importants ont été apportés au budget ou à la distribution des fonds par rapport à la proposition originale et soumettez-la à nouveau avec le formulaire Profil du stagiaire.}

\end{document}
%% :folding=explicit:wrap=soft: