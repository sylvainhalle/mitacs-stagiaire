\newcommand{\addul}[1]{\underline{#1}}
\titleformat{\section}{\centering\Large\bfseries}{\thesection}{1em}{\addul}
\stepcounter{section}
\section*{4. Protocole d'entente Mitacs}
{
\bfseries
Les participantes et participants dont le nom apparaît ci-dessous confirment que les renseignements présentés sont exacts et qu'ils reflètent leur intention de déposer une demande au programme Mitacs Accélération. Ils et elles acceptent également d'effectuer un stage portant sur la demande ci-jointe. Les participantes et participants reconnaissent avoir lu, compris et accepté d'observer et de soutenir les responsabilités du projet qui les visent, lesquelles peuvent être consultées au \url{https://www.mitacs.ca/fr/programmes/acceleration/responsabilites-projet} et comprennent, sans s'y limiter, ce qui suit : il est entendu que la contribution de l'organisation partenaire sera versée à Mitacs inc. en dollars canadiens avant le début du stage; dans l'éventualité où la contribution de l'organisation partenaire est à l'établissement d'enseignement, l'établissement d'enseignement fera suivre ces fonds à Mitacs. À la suite de l'approbation de la recherche du projet et de la réception des fonds de l'organisation partenaire par Mitacs, Mitacs versera les fonds à l'établissement d'enseignement canadien à titre de subvention de recherche à la professeure superviseure canadienne ou au professeur superviseur canadien et une allocation/un salaire sera versé à l'étudiant·e par l'établissement d'enseignement à partir de cette subvention. Les dépenses associées à cette demande, telles que décrites dans le budget, doivent seulement être engagées après l'approbation de la recherche et la réception des fonds de l'organisation partenaire par Mitacs.

Mitacs n'assume aucune responsabilité à l'égard de toute perte, y compris, sans s'y limiter, à des accidents, des maladies, des déplacements ou autres pertes qui peuvent survenir pendant la période de stage. Chacune des parties signataires convient qu'il lui incombe de s'assurer qu'elle dispose d'une assurance appropriée et qu'elle répond aux politiques institutionnelles concernant les exigences en matière de santé et de sécurité ainsi que toute autre préparation requise avant d'entreprendre un voyage. Les parties conviennent également que la ou le stagiaire devra produire un rapport de fin de projet et que tous les participants répondront à un sondage de fin de projet qui sera remis à Mitacs au plus tard un mois après la fin du projet.

\textit{Pour les projets qui comprennent un déplacement international:} En reconnaissant que les séjours à l'étranger peuvent grandement enrichir les connaissances et l'acquisition d'expérience d'un stagiaire, Mitacs approuvera les déplacements à l'étranger sous réserve que ceux-ci ne mettent pas en danger la sécurité des stagiaires et que les politiques de l'établissement d'enseignement d'attache soient respectées. En signant le présent accord, vous reconnaissez que l'établissement d'enseignement d'attache accepte de s'engager à aider la ou le stagiaire dans ses démarches visant à satisfaire toutes les exigences de l'établissement d'enseignement ayant trait à la recherche à l'étranger et que la ou le stagiaire comprend sa responsabilité de souscrire une assurance maladie appropriée pour sa destination. Les participants aux projets qui comprennent des déplacements internationaux reconnaissent que des responsabilités du projet supplémentaires s'appliquent à chacun d'eux qui peuvent être consultées au \url{https://www.mitacs.ca/fr-ca/nos-programmes/acceleration/}. Les participants à des projets qui comprennent des déplacements internationaux reconnaissent également que Mitacs sera dans l'impossibilité de verser les fonds et que le stage ne pourra pas commencer tant que le formulaire prédépart international et le code de déontologie signés n'auront pas été reçus.

Toutes les parties participant au programme Mitacs Accélération sont tenues de respecter les règlements standards sur la propriété intellectuelle (PI) établis par l'établissement d'enseignement où la ou le stagiaire est inscrit·e, à moins qu'un accord séparé, valide pendant la durée du stage, soit négocié entre le ou les établissements d'enseignement et l'organisation partenaire. Si vous avez des ententes séparées relatives à la PI entre vous et l'établissement d'enseignement, vous reconnaissez en signant ce protocole d'entente être assujetti·e à leurs conditions précises.  Dans le cas où vous n'avez aucune entente séparée, vous êtes assujetti·e aux conditions standards de PI de l'établissement d'enseignement et en signant cette entente, vous acceptez les conditions de l'établissement d'enseignement où la ou le stagiaire est inscrit·e. Les politiques en matière de PI propres à chaque établissement en ce qui concerne les stages Accélération se trouvent à la page \url{https://www.mitacs.ca/fr/programmes/acceleration/faq}.

Les participants acceptent également que Mitacs affiche le titre du projet, l'aperçu du projet pour le public, le nom du ou des organisations partenaires, le nom de la ou du stagiaire ou des stagiaires, le nom de la ou du superviseur·e ou des superviseur·es et l'établissement d'enseignement participant sur la page www.mitacs.ca/fr/projets. Ces renseignements pourraient aussi être utilisés par Mitacs pour faire la publicité du programme Mitacs Accélération. La Politique de confidentialité de Mitacs est disponible sur la page suivante : \url{https://www.mitacs.ca/fr/declaration-de-confidentialite}.

Les participantes et participants au stage (stagiaire, professeur·e superviseur·e et organisation partenaire) acceptent aussi le ou les addendas suivants:

Mitacs ne requiert, ne vérifie, ni n'impose aucune autre condition que celles indiquées par les participantes et participants dans le ou les addendas ci-dessus.
}

%%:wrap=soft: